% Appendix A

\chapter{Glosario} % Main appendix title
\label{AppendixA} % For referencing this appendix elsewhere, use \ref{AppendixA}

La Tabla \ref{tab:glosario} muestra la lista de acrónimos y términos utilizados en el presente trabajo y busca ser una referencia rápida para el lector.


\begin{table}[h]
	\centering
	\caption[Tipos de emuladores]{Glosario de acrónimos.}
	\begin{tabular}{p{.15\textwidth} p{.84\textwidth}}
		\toprule
		\textbf{Acrónimo} & \textbf{Significado} \\
		\midrule
    \textbf{API} & Interfaz de programación de aplicaciones (\textit{Application Programming Interface}). \\
    \textbf{CAN} & Red de área de control (\textit{Controller Area Network}). \\
    \textbf{CPU} & Unidad central de procesamiento (\textit{Central Processing Unit}). \\
    \textbf{CSV} & Valores separados por comas (\textit{Comma Separated Values}). \\
    \textbf{FSW} & Software de vuelo (\textit{Flight Software}). \\
    \textbf{MCU} & Unidad de control de memoria (\textit{Memory Control Unit}). \\
    \textbf{OBC} & Computadora a bordo (\textit{On Board Computer}). \\
    \textbf{OBSW} & Software a bordo (\textit{On Board Software}) \\
    \textbf{PROM} & Memoria programable de solo lectura (\textit{Programmable Read-Only Memory}). \\
    \textbf{PC} & Contador de programa (\textit{Programm Counter}). \\
    \textbf{RAM} & Memoria de acceso aleatorio (\textit{Random Access Memory}). \\
    \textbf{UART} & Receptor transmisor asíncrono universal (\textit{Universal Asynchronous Receiver Transmitter}). \\
		\bottomrule
		\hline
	\end{tabular}
	\label{tab:glosario}
\end{table}
