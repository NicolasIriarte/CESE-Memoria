\chapter{Introducción específica} % Main chapter title
\label{Chapter2}

El presente capítulo describe los elementos componentes del sistema, el principio de funcionamiento y los requerimientos acordados con el cliente sobre el sistema.



%----------------------------------------------------------------------------------------
%	SECTION 1
%----------------------------------------------------------------------------------------


\section{Elementos componentes del sistema}
\label{sec:elementos_componentes_sistema}

Para entender el funcionamiento del sistema, es necesario conocer los elementos que lo componen. La presente sección describirá tanto los elementos físicos necesarios para conocer al sistema que se busca representar, como las herramientas de software utilizadas para el desarrollo del emulador.

\subsection{Elementos físicos del sistema}
\label{subsec:elementos_fisicos}

La presente sección dará una descripción general de los elementos que serán de vital importancia para el proyecto.

\subsubsection{Bootloader}
\label{subsec:bootloader}

El \textit{bootloader} es una pieza de software relativamente simple, cuyo principal objetivo es cargar el software de vuelo en la memoria RAM del sistema. Una vez cargado el software de vuelo, le cede el control al software de vuelo, que se encarga de ejecutar la misión del sistema.

En misiones espaciales es común que el \textit{bootloader} tenga la capacidad de verificar la integridad del software de vuelo antes de cargarlo en la memoria RAM, asimismo, de cargar diferentes versiones del software de vuelo en caso de que alguna de ellas se encuentre dañada.

\subsubsection{Firmware}
\label{subsec:firmware}

El \textit{firmware} es un software que se encuentra almacenado en la memoria no volátil del sistema, encargado de inicializar sus periféricos. Usualmente, este componente es desarrollado por un equipo especializado tanto en la misión para la que se está desarrollando el sistema, como en la arquitectura del microprocesador que se está utilizando. En el caso de aplicaciones espaciales, el \textit{firmware} es llamado software de vuelo o \textit{Fly Software} (FSW).

Normalmente, en la memoria no volátil del sistema se almacenan múltiples versiones del software de vuelo. Varias de ellas están presentes para servir de redundancia, en caso de que alguna de ellas se encuentre dañada, como para almacenar distintas versiones del software de vuelo con modos de operación más limitados pero seguros.

\subsubsection{Unidad de procesamiento}
\label{subsec:unidad_procesamiento}
Todo lo previamente mencionado se ejecuta en la unidad de procesamiento o CPU. La CPU es el componente encargado de ejecutar las instrucciones tanto del \textit{bootloader} como del software de vuelo.

El CPU interactúa con los periféricos del sistema, especialmente con la memoria RAM del sistema, ya que es el medio de almacenamiento masivo de alta velocidad de acceso.

El CPU, una vez encendido, empezará a leer las instrucciones de una posición dada por el fabricante y ejecutarlas secuencialmente. Es deber del desarrollador posicionar al \textit{bootloader} en esa posición de memoria.

\subsection{Software utilizado}
\label{subsec:software_utilizado}

La presente sección describirá las herramientas de software que fueron utilizadas para el desarrollo del trabajo.

\subsubsection{Lenguaje de programación C++}
\label{subsec:cpp}

Dado que el emulador es un software que simula el comportamiento de un CPU, es necesario que el lenguaje de programación utilizado sea de bajo nivel, permitiendo mejores tiempos de ejecución y un mayor control sobre los recursos utilizados. Por lo tanto, se decidió utilizar C++ como lenguaje de programación para el desarrollo del emulador.

Otro motivo para su elección es que C++ es un lenguaje de programación ampliamente utilizado en la industria, por lo que facilita la comunicación con otros desarrolladores y la integración con otros sistemas \citep{NASA_GITHUB}.

\subsubsection{Herramientas Clang}
\label{subsec:clang_tidy}

Se utilizó Clang Tidy \citep{CLANG_TIDY} como herramienta de análisis estático de código. Permitiendo una mayor calidad del código y sintaxis uniforme. Asimismo, se forzaron reglas para evitar malas prácticas comunes en los lenguajes de programación de bajo nivel, tales como el uso de memoria no inicializada, o aritmética de punteros.

También se utilizó Clang Format  \citep{CLANG_FORMAT} para mantener un estilo de código uniforme en todo el proyecto.


\subsubsection{Editor de texto Emacs}
\label{subsec:emacs}

Se utilizó el editor de texto Emacs \citep{EMACS} como herramienta de desarrollo tanto del código de programación como para toda su documentación.

Se eligió Emacs debido a la familiaridad con la herramienta, y a la facilidad de integración con otras herramientas de desarrollo \citep{MELPA}.


\subsubsection{Sistema de composición de textos LaTeX}
\label{subsec:latex}

Se utilizó LaTeX \citep{LATEX} como herramienta de escritura de la documentación para el manual de usuario del proyecto. Dicha elección se tomo debido a que LaTeX es una herramienta ampliamente utilizada en la industria para la escritura de documentos técnicos.

Para la edición de los documentos se utilizó el editor de texto Emacs con el modo de edición AucTeX \citep{AUCTEX}.

El manual de usuario busca ser una guía completa y en detalle de la herramienta, no solo explicando la interfaz expuesta, sino decisiones de diseño y funcionamiento interno de la herramienta.

\subsubsection{Doxygen}
\label{subsec:doxygen}

Se utilizó Doxygen \citep{DOXYGEN} como herramienta de generación de documentación del código fuente. Con esta herramienta se generó un sitio web que describe la interfaz para interactuar con la herramienta (API). Dicha documentación busca ser una guía rápida para desarrolladores que estén utilizando el emulador.

\subsubsection{Gitlab}
\label{subsec:gitlab}

Se utilizó Gitlab \citep{GITLAB} como herramienta de control de versiones para el desarrollo del proyecto. Se optó por esta herramienta debido a que permite una integración sencilla con otras herramientas de desarrollo y familiaridad con la herramienta.

Cabe aclarar que se utilizó una instancia privada de Gitlab, para mantener la privacidad del código fuente del proyecto.


\section{Principio de funcionamiento}
\label{sec:principio_funcionamiento}

El funcionamiento de un CPU, y por lo tanto de un emulador, se puede mostrar de manera simplificada de forma gráfica en la figura \ref{fig:functionamiento_emulador}.

\begin{figure}[htbp]
	\centering
	\includegraphics[width=1\textwidth]{./Figures/funcionamiento_emulador}
	\caption{Funcionamiento de alto nivel de un CPU o emulador.}
	\label{fig:functionamiento_emulador}
\end{figure}

Tal como se observa en el gráfico, el procesador o CPU lee instrucciones de la memoria RAM y procede a interpretarla. En la arquitectura seleccionada, todas las instrucciones son de 32 bits, por lo que todas las lecturas de memoria son idénticas. La posición de memoria a leer es determinada por un registro especial llamado \textit{Program Counter} (PC por sus siglas en inglés) ubicado en el estado del CPU.

Una vez leída la instrucción, se la decodifica para saber de cual se trata, y ejecutar la acción necesaria. Las instrucciones pueden ser de distintos tipos, como aritméticas, de salto condicional o incondicional, y afectan al estado del CPU de diferentes formas.

Dicho flujo se sucede de manera continua y las instrucciones no siempre se ejecutan en el orden en el que se encuentran en la memoria. Esto se debe a que ciertas instrucciones pueden modificar el valor del PC, y por lo tanto, la próxima instrucción a ejecutar.

Algo a destacar, es que la arquitectura que se desarrolló es \textit{big-endian}, es decir, los bytes más significativos se encuentran en las direcciones de memoria más altas. Esto es importante ya que la arquitectura x86, en la que se desarrolló el emulador, es \textit{little-endian}, por lo que se deberán realizar conversiones de endianess en cada lectura o escritura de memoria.

Este detalle es de vital importancia ya que un mismo valor de memoria puede ser interpretado de manera diferente dependiendo de la arquitectura que se esté utilizando. La figura \ref{fig:endianess} muestra un ejemplo de cómo se interpretaría el valor \texttt{0x12345678} en una arquitectura \textit{big-endian} y en una \textit{little-endian}.

\begin{figure}[htbp]
	\centering
	\includegraphics[width=0.85\textwidth]{./Figures/endianess}
	\caption{Relación entre valores y su representación en memoria.}
	\label{fig:endianess}
\end{figure}

Hay que recordar que aunque los bytes estén en distintos ordenes en memoria, el valor que representan es exactamente el mismo.


\section{Requerimientos}
\label{sec:requerimientos}

A continuación, se presentan los requerimientos funcionales y no funcionales acordados con el cliente.

\subsection{Requerimientos funcionales}
\label{subsec:requerimientos_funcionales}

\begin{enumerate}
\item El emulador deberá ejecutar los mismos binarios que se utilizan en el hardware real.
\item El sistema debe ser compatible con el sistema operativo Linux.
\item El emulador deberá poder ejecutar correctamente parte del set de instrucciones del procesador real.
\item Se deberán desarrollar tests unitarios que verifiquen el set de instrucciones desarrollado.
\item Se deberá desarrollar un ambiente de automatización de pruebas en Gitlab CI.
\item El software podrá utilizarse como biblioteca compartida.
\end{enumerate}

\subsection{Requerimientos no funcionales}
\label{subsec:requerimientos_no_funcionales}

\begin{enumerate}
\item El software debe mantenerse bajo control de versiones en Gitlab.
\item El software deberá ser escrito en C++.
\item La API expuesta deberá estar documentada con Doxygen.
\item La API expuesta deberá ser en C.
\item Se realizará un manual de usuario que describa los funcionamientos clave del software.
\end{enumerate}

